\documentclass[xcolor=dvipsnames]{beamer}
\usepackage[utf8]{inputenc}
\usepackage[T1]{fontenc}
\usepackage{fixltx2e}
\usepackage{graphicx}
\usepackage{longtable}
\usepackage{float}
\usepackage{wrapfig}
\usepackage{soul}
\usepackage{textcomp}
\usepackage{marvosym}
\usepackage{wasysym}
\usepackage{latexsym}
\usepackage{amssymb}
\usepackage{hyperref}
\usepackage{cancel}
\usepackage[percent]{overpic}
\usepackage{listings}
\usepackage{color}
\lstset{ %
  language=Python,
  basicstyle=\ttfamily\tiny,
  emphstyle=\color{red},
  keywordstyle=\color{black}\bfseries,
  identifierstyle=\color{DarkOrchid}\ttfamily,
  commentstyle=\color{Brown}\rmfamily\itshape,
  stringstyle=\color{blue}\slshape,
  showstringspaces=false,
  frame=single,                   % adds a frame around the code
}

\setbeamertemplate{navigation symbols}{}
\useoutertheme{infolines}
\usecolortheme[named=violet]{structure}
\setbeamertemplate{items}[circle]

\DeclareGraphicsExtensions{.pdf,.png,.jpg}
\title[Data Handling]{CCD Acceptance Testing Data Handling}
\author{Brett Viren and Tom Throwe}
\institute[BNL]
{
  Physics Department

  \includegraphics[height=1.5cm]{bnl-logo}
}

\date{\today}

\definecolor{rootpink}{RGB}{255,0,255}
\definecolor{macroyellow}{RGB}{255,215,0}

\hypersetup{
  pdfkeywords={},
  pdfsubject={},
  pdfcreator={pdflatex, beamer and emacs the digital blood, sweat and tears}}

\begin{document}

\maketitle

\begin{frame}
\frametitle{Outline}
\tableofcontents
\end{frame}

\section{Overview}

\begin{frame}[fragile]
  \frametitle{High Level Data Flow}
  \includegraphics[width=\textwidth]{dataflow}
  \begin{enumerate}
  \item Test stations and analysis codes produces result files.
  \item All result files are stored permanently to a well defined disk
    location.
    \begin{itemize}
    \item Portability via rooting at \texttt{\$CCDTEST\_ROOT}
    \end{itemize}
  \item Specially formatted result summary files are validated, parsed
    and their contents are uploaded to the database.
  \end{enumerate}

  \footnotesize
  By using intermediate files in a common, popular format we can make
  a loose coupling between the test stations and analyzers and the
  database.  This will make supporting the varied test platforms and
  environments simpler.
\end{frame}

\section{Job Harness}

\begin{frame}[fragile]
  \frametitle{Job Harness - Version Control} 

  Assure the capture of software versions, input parameters, runtime
  environment and output results.

  \begin{columns}
    \begin{column}{0.5\paperwidth}
      \begin{itemize}
      \item Tests run by a common job harness (GNU/Linux and Windows).
      \item Record input parameters.
      \item All software versions recorded by (GIT) commit hash.
      \item Invokes the test software.
      \item Post-processing if needed to produce required interchange
        file format.
      \end{itemize}
    \end{column}
    \begin{column}{0.5\paperwidth}
      \includegraphics[width=\textwidth]{job-harness}
    \end{column}
  \end{columns}

  \vspace{2mm}

  $\rightarrow$Currently at the conceptual level.
\end{frame}

\section{Test Result Files}

\begin{frame}
  \frametitle{Station Results}

  Station or analysis software produces result files in three categories:

  \begin{columns}
    \begin{column}{0.5\paperwidth}
      \footnotesize
      \begin{description}
      \item[Metadata FITS] Describes test software and enumerates results
        files and any auxiliary files.  (well defined schema)
      \item[Result FITS] Result summary files to be parsed and uploaded
        into the DB. (well defined schema)
      \item[Auxiliary] Any additional files to archive. These may
        contain some info that is redundant to the above (arbitrary
        formats)
      \end{description}
    \end{column}
    \begin{column}{0.4\paperwidth}
      \includegraphics[width=\textwidth]{files}
    \end{column}
  \end{columns}
  All files are archived, first two types are parsed into the database.

\end{frame}

\begin{frame}
  \frametitle{Metadata FITS File} 

  Collects information about all result files.  Here a ``test'' means
  a run on a test station or an offline analysis.  

  Contains 4 HDUs:

  \begin{enumerate}
  \item Primary HDU, header only, (this will likely expand)
    \begin{description}
    \item[\texttt{TESTNAME}] a canonical name for the test.
    \item[\texttt{DATE-OBS}] UTC time stamp for when the test was run.
    \item[\texttt{USERNAME}] Name for the test station operator or analyzer.
    \end{description}
  \item Test's \textbf{software description} consisting of list of main
    programs and their GIT SHA1 commit hash and tag.  Lists are stored
    in a FITS table.
  \item List of \textbf{Result FITS files} to be parsed into the database (see next).
    Stored as a table of filenames and content SHA1 digest hashes.
  \item List of all \textbf{auxiliary files} to be referenced from the
    database.  Lists stored in same form as \#3.
  \end{enumerate}

\end{frame}


\begin{frame}
  \frametitle{Result FITS File}
  Collects all result data that will be parsed into the database.

  \vspace{2mm}

  Skeleton is simple:
  \begin{enumerate}
  \item Common PrimaryHDU for all tests.
    \begin{description}
    \item[\texttt{TESTNAME}] Canonical name for the test as a whole.
    \item[\texttt{EXTNAME}] the HDU schema name
    \item[\texttt{EXTVER}] this HDU's schema version
    \end{description}
  \item Test-specific secondary HDUs with some common cards:
    \begin{description}
    \item[\texttt{EXTNAME}] the HDU schema name
    \item[\texttt{EXTVER}] this HDU's schema version
    \item[...] all additional header cards and/or table columns needed
      to hold part of a result
    \end{description}
  \item Etc, as many secondary HDUs as needed to hold a test's result. 
    \begin{itemize}
    \item Each HDU follows a well defined schema.
    \end{itemize}
  \end{enumerate}

  The schema for each station/analysis result extends this base as
  needed.
\end{frame}



\section{Schema Representation In Python}

\begin{frame}
  \frametitle{Schema Definition in Python}

  Implementing a FITS schema in Python

  \vspace{5mm}

  \begin{enumerate}
  \item Subclass \texttt{pyfits} HDU classes to 
    \begin{itemize}
    \item[$\Rightarrow$] achieve valid low-level FITS format, by construction.
    \end{itemize}
  \item Add a \texttt{validate()} method.
    \begin{itemize}
    \item Low-level, common validation in intermediate subclasses
    \item Additional validation in per-test, leaf subclasses
    \end{itemize}
  \item One Python module for each test result
    \begin{itemize}
    \item  Schema defined as hard-coded list of classes
    \end{itemize}
  \item Replace \texttt{pyfits.open()} to allow reading in and validating compliant files.
  \end{enumerate}

\end{frame}

\begin{frame}
  \frametitle{Current Status of Schema Implementation}

  Status:
  \begin{itemize}
  \item Basic skeleton exists.
  \item Unit tests written with partial coverage.
  \item Metadata file schema implemented.
  \item Jim Frank's photon transfer curve linearity analysis results partially
    implemented
  \item Some basic validation code written for the existing schema. 
  \item Documentation temporarily at:\\
    {\footnotesize
    \url{http://www.phy.bnl.gov/~bviren/lsst/lcatr/doc/lcatr/build/html/}}
  \item Code temporarily at:\\
    {\footnotesize
    \url{https://github.com/brettviren/lcatr}}
  \end{itemize}

\end{frame}

\section{A Bit About Git}

\begin{frame}
  \frametitle{Usage of Git repositories}
  
  \begin{itemize}
  \item A set of git repositories local to the testing institutions will be used.
    \begin{itemize}
    \item RACF setup mostly complete, provide central repos
    \item Other tests sites will clone.
    \item Anticipate cloning to SLAC's Git area for archive.
    \end{itemize}
  \item All test station and analysis code will be in git repositories
    \begin{itemize}
    \item Git commit hash used to indelibly mark the versions used.
    \item Issues with ever-changing files (eg Mathematica) need addressing.
    \item Any proprietary code will have its binaries committed.
    \end{itemize}
  \item High level data handling code.
    \begin{itemize}
    \item Schema validation, database and web application, job harness.
    \item Currently at github.com, will move to RACF.
    \end{itemize}
  \end{itemize}
  
\end{frame}

\section{To Do}

\begin{frame}
  \frametitle{To Do}
  \begin{itemize}
  \item Still understanding the details of the data schema covering
    the various test stations and analyses.
    \begin{itemize}
    \item Not practical to use the Python schema directly in all cases.
    \item Will develop conversion/summation scripts in these cases.
    \end{itemize}
  \item Schema evolution handling needs implementing.
  \item Job harness system at the concept stage.
    \begin{itemize}
    \item Will leverage GIT and maybe use modules.sf.net to
      set/capture environment.
    \end{itemize}
  \item Still fuzzy on:
    \begin{itemize}
    \item How to handle calibration information?
    \item How to handle environment monitoring data?
    \end{itemize}
  \end{itemize}
\end{frame}

\end{document}

